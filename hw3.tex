\documentclass{article}
\usepackage[utf8]{inputenc}
\usepackage[T2A]{fontenc}
\usepackage[utf8]{inputenc}
\usepackage[russian]{babel}
\usepackage[margin=3cm]{geometry}
\usepackage{paralist}
\usepackage{amsthm, amsmath, amsfonts, amssymb}
\usepackage{mathtools} % \mathclap
\usepackage{bm}
\usepackage{dsfont}
\usepackage{hyperref}
\usepackage{graphicx}
\usepackage{multirow}
\usepackage{comment}
\usepackage{xcolor, colortbl}
\usepackage{xifthen, xspace}
\usepackage{caption, subcaption}
\usepackage{lscape}
\usepackage{braket}
\usepackage{epigraph}
\usepackage{sectsty}
\usepackage{listings}

\hypersetup{
    colorlinks=true,
    linkcolor=blue,
    filecolor=magenta,      
    urlcolor=blue,
    pdftitle={Overleaf Example},
    pdfpagemode=FullScreen,
    }

\title{Теоретические модели вычислений \\
ДЗ №3: Машины Тьюринга и квантовые вычисления}

\date{1 мая 2022 года}

\begin{document}

\maketitle

\section{Введение}

Для того, чтобы работа была принята, требуется соблюдать ряд условий:

\begin{enumerate}
    \item Текст решения домашней работы должен быть подготовлен в одном из следующих форматов: \href{https://www.latex-project.org/}{\LaTeX}, \href{https://en.wikipedia.org/wiki/Markdown}{Markdown}, \href{https://asciidoc-py.github.io/index.html}{AsciiDOC}
    \item Дополнительно должен быть приложен PDF файл
    \item Для набора рекомендуется использовать сервисы типа \href{https://www.overleaf.com}{Overleaf}
    \item Работу можно выполнить на английском языке
    \item Иллюстрации, должны быть описаны на декларативном языке \href{https://graphviz.org/}{Graphviz} или \href{https://mermaid-js.github.io/mermaid/#/}{Mermaid}. В сам документ можно вставить отрендеренную картинку.
    \item За каждое задание можно получить определённое количество количество баллов, указанное рядом с заданием
    \item Мягкий дедлайн -- 20 дней (12:00 20 мая 2022 года), после него $-30\%$ к оценке
    \item Для зачёта надо набрать $60\%$ от максимальной оценки
    \item Максимальное количество баллов: 10.
    \item Работа должна быть выложена на \href{https://github.com/}{GitHub}, в репозитории, созданном к ней
\end{enumerate}

\section{Машины Тьюринга}

Работу требуется выполнять в системе \url{turingmachine.io}. \\\\
Для сдачи заданий 1-2 требуется прикрепить файлы YAML с исходным кодом проекта. Каждый файлы должен иметь наименование задание\_пункт.yml, к примеру 1\_1.yml для первой задачи первого задания. \\\\

\subsection{Операции с числами}

Реализуйте машины Тьюринга, которые позволяют выполнять следующие операции:
\begin{enumerate}
    \item Сложение двух унарных чисел (1 балла)
    \item Умножение унарных чисел (1 балл)
\end{enumerate}


\subsection{Операции с языками и символами}

Реализуйте машины Тьюринга, которые позволяют выполнять следующие операции:
\begin{enumerate}
    \item Принадлежность к языку $L = \{ 0^n1^n2^n \}, n \ge 0$ (0.5 балла)
    \item Проверка соблюдения правильности скобок в строке (минимум 3 вида скобок) (0.5 балла)
    \item Поиск минимального по длине слова в строке (слова состоят из символов 1 и 0 и разделены пробелом) (1 балл)
\end{enumerate}


\section{Квантовые вычисления}

Для выполнения заданий по квантовым вычислениям требуется QDK. Его можно скачать здесь: \url{https://docs.microsoft.com/en-us/azure/quantum/install-overview-qdk}. 
\\\\
Но можно использовать любой пакет, типа \url{https://qiskit.org/}. 
\\\\
В качестве решения задачи надо предоставить схему алгоритма для частного случая при фиксированном количестве кубитов и фиксированных состояниях. 


\subsection{Генерация суперпозиций 1 (1 балл)}

Дано $N$ кубитов ($1 \le N \le 8$) в нулевом состоянии $\Ket{0\dots0}$. Также дана некоторая последовательность битов, которое задаёт ненулевое базисное состояние размера $N$. Задача получить суперпозицию нулевого состояния и заданного.

$$\Ket{S} = \frac{1}{\sqrt2}(\Ket{0\dots0} +\Ket{\psi})$$

То есть требуется реализовать операцию, которая принимает на вход:

\begin{enumerate}
    \item Массив кубитов $q_s$
    \item Массив битов $bits$ описывающих некоторое состояние $\Ket{\psi}$. Это массив имеет тот же самый размер, что и $qs$. Первый элемент этого массива равен $1$.
\end{enumerate}


Заготовка для кода:
\begin{lstlisting}
namespace Solution {
        open Microsoft.Quantum.Primitive;
        open Microsoft.Quantum.Canon;
        operation Solve (qs : Qubit[], bits : Bool[]) : ()
        {
            body
            {

            }
        }
}
\end{lstlisting}


\subsection{Различение состояний 1 (1 балл)}

Дано $N$ кубитов ($1 \le N \le 8$), которые могут быть в одном из двух состояний:

$$\Ket{GHZ} = \frac{1}{\sqrt2}(\Ket{0\dots0} +\Ket{1\dots1})$$
$$\Ket{W} = \frac{1}{\sqrt N}(\Ket{10\dots00}+\Ket{01\dots00} + \dots +\Ket{00\dots01})$$

Требуется выполнить необходимые преобразования, чтобы точно различить эти два состояния. Возвращать $0$, если первое состояние и 1, если второе. 
\\\\
Заготовка для кода:
\begin{lstlisting}
namespace Solution {
        open Microsoft.Quantum.Primitive;
        open Microsoft.Quantum.Canon;
        operation Solve (qs : Qubit[]) : Int
        {
            body
            {

                return 
            }
        }
}
\end{lstlisting}


\subsection{Различение состояний 2 (2 балла)}

Дано $2$ кубита, которые могут быть в одном из двух состояний:

$$\Ket{S_0} = \frac{1}{2}(\Ket{00} + \Ket{01} + \Ket{10} + \Ket{11})$$
$$\Ket{S_1} = \frac{1}{2}(\Ket{00} - \Ket{01} + \Ket{10} - \Ket{11})$$
$$\Ket{S_2} = \frac{1}{2}(\Ket{00} + \Ket{01} - \Ket{10} - \Ket{11})$$
$$\Ket{S_3} = \frac{1}{2}(\Ket{00} - \Ket{01} - \Ket{10} + \Ket{11})$$


Требуется выполнить необходимые преобразования, чтобы точно различить эти четыре состояния. Возвращать требуется индекс состояния (от $0$ до $3$). 
\\\\
Заготовка для кода:
\begin{lstlisting}
namespace Solution {
        open Microsoft.Quantum.Primitive;
        open Microsoft.Quantum.Canon;
        operation Solve (qs : Qubit[]) : Int
        {
            body
            {

                return 
            }
        }
}
\end{lstlisting}


\subsection{Написание оракула 1 (2 балла)}

Требуется реализовать квантовый оракул на $N$ кубитах ($1 \le N \le 8$), который реализует следующую функцию: $f(\pmb{x}) = (\pmb{b}\pmb{x}) \mod 2$, где  $\pmb{b} \in \{0,1\}^N$ вектор битов и  $\pmb{x}$ вектор кубитов. Выход функции записать в кубит $\pmb{y}$. Количество кубитов $N$ ($1 \le N \le 8$). 
\\\\
Заготовка для кода:
\begin{lstlisting}
namespace Solution {
        open Microsoft.Quantum.Primitive;
        open Microsoft.Quantum.Canon;
        operation Solve (x : Qubit[], y : Qubit, b : Int[]) : ()
        {
            body
            {

            }
        }
}
\end{lstlisting}

\end{document}